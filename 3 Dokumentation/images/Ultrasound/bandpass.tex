%\title{18W MOSFET amplifier)
% Title: 18W MOSFET amplifier, with npn transistor.
% Author: Ramón Jaramillo.
% Source: http://www.circuitstoday.com/mosfet-amplifier-circuits
% Abstract:
% This example made use of circuitikz and siunits packages for drawing a 18W MOSFET  Amplifier for one-channel. It is mandatory than siunitx and related packages have been instaled, acording to your LaTeX distribution.
\documentclass[10pt,a4paper]{standalone}
\usepackage{filecontents,pgfplots}

\begin{filecontents}{pistonkinetics.dat}
freq    amplitude  verstäkrung
0.1000  890 		 7.03279978
0.5000  450			12.95634964
1.0000  430			13.3512308
1.5000  440			13.15154638
2.0000  440			13.15154638
2.5000  450			12.95634964
3.0000  470			12.57864275
3.5000  520			11.70053304
4.0000  570			10.9031028
4.5000  680			 9.370421659
5.0000  590			10.60355968
5.5000  530			11.53508252
6.0000  500			12.04119983
6.5000  500			12.04119983
7.0000  500			12.04119983
7.5000  530			11.53508252
8.0000  570			10.9031028
8.5000  610			10.31400321
9.0000  650			 9.76233278
9.5000  760			 8.404328068
10.0000  880		 7.13094647
\end{filecontents}

\begin{document}
\pgfplotstableread{pistonkinetics.dat}{\pistonkinetics}
\begin{tikzpicture}[scale=1]
\begin{axis}[minor tick num=1,
xlabel=Frequenz in MHz, 
%ylabel=Verstärkung in dB]
ylabel=Eingangsspannung $U_{in}$ in Vpp]
\addplot [black,very thick] table [x={freq}, y={amplitude}] {\pistonkinetics};
%\addplot [dashed,red,very thick] table [x={freq}, y={amplitude}] {\pistonkinetics};
%\addplot [dashed,blue,very thick] table [x={freq}, y={amplitude}] {\pistonkinetics};
\end{axis}
\end{tikzpicture}
\end{document}