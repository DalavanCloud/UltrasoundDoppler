\chapter{Abkürzungen und Begriffsdefinition}

\section{Abkürzungen} 
\begin{acronym}[mosfetmosfet]\itemsep0pt
\acro{adc}[ADC]{Analog-Digital Converter}
%\acro{alu}[ALU]{Arithmetic logic unit}
\acro{api}[API]{Application programming interface}
\acro{bom}[BOM]{Bill of Materials}
\acro{bzw}[bzw.]{beziehungsweise}
\acro{ca}[ca.]{circa}
%\acro{cad}[CAD]{Computer Aided Design}
\acro{cmsis}[CMSIS]{Cortex Microcontroller Software Interface Standard}
\acro{cw}[CW]{Continuous Wave}
\acro{cpld}[CPLD]{Complex Programmable Logic Device}
\acro{db}[dB]{Dezibel}
\acro{dac}[DAC]{Digital-Analog Converter}
\acro{dc}[DC]{Direct Current}
\acro{dft}[DFT]{Diskrete Fourier Transformation}
\acro{esb}[ESB]{Ersatzschaltbild}
%\acro{eeprom}[EEPROM]{Electrically Erasable Programmable Read Only Memory}
\acro{dh}[d.h.]{das heißt}
\acro{dma}[DMA]{direct memory access}
%\acro{dnf}[DNF]{disjunktive Normalform}
\acro{emi}[EMI]{Electromagnetic interference}
\acro{fft}[FFT]{Fast Fourier Transformation}
\acro{fifo}[FIFO]{First In, First Out}
\acro{fir}[FIR]{finite impulse response}
\acro{fpga}[FPGA]{Field Programmable Gate Array}
%\acro{fps}[FPS]{frames per second}
\acro{fpu}[FPU]{floating-point unit}
\acro{ghz}[GHz]{Giga Herz}
\acro{gui}[GUI]{Graphical User Interface}
\acro{hal}[HAL]{Hardware Abstraction Layer}
\acro{hf}[HF]{Hochfrequenz}
%\acro{hdk}[HDK]{Hardware Development Kit}
\acro{hdl}[HDL]{Hardware Description Language}
%\acro{hkt}[Hkt]{Hämatokrit}
\acro{ic}[IC]{Integrated Circuit}
\acro{ide}[IDE]{integrated development environment}
\acro{iir}[IIR]{infinite impulse response}
%\acro{isbn}[ISBN]{Internationale StandardBuchNummer}
%\acro{ito}[ITO]{Indium-Tin-Oxide}
\acro{khz}[kHz]{Kilohertz}
%\acro{knf}[KNF]{konjunktive Normalform}
\acro{ldo}[LDO]{Low-dropout regulator}
\acro{lna}[LNA]{Low Noise Amplifier}
\acro{lut}[LUT]{Look-up Table}
\acro{ma}[mA]{Milliampere}
\acro{mcu}[MCU]{Microcontroller Unit}
\acro{mhz}[MHz]{Megahertz}
%\acro{mcv}[MCV]{mittleres corpusculäres Volumen der Erythrozyten}
\acro{mosfet}[MOSFET]{metal oxide semiconductor field-effect transistor}
\acro{msps}[MSps]{Megasamples per second}
\acro{nf}[NF]{Niederfrequenz}
\acro{oop}[OOP]{Objektorientierte Programmierung}
\acro{pc}[PC]{Personal Computer}
\acro{pcb}[PCB]{Printed Circuit Board}
%\acro{pla}[PLA]{Programmable Logic Array}
\acro{pp}[pp]{peak-peak}
\acro{prf}[PRF]{Pulse Repetition Frequency}
\acro{pw}[PW]{Pulsed Wave}
%\acro{qfn}[QFN]{Quad-Flat-No-leads (Package)}
\acro{ram}[RAM]{Random-Access Memory}
\acro{rd}[R\&D]{research and development}
\acro{rf}[RF]{Radio Frequenz}
\acro{resp}[resp.]{respektive}
\acro{roi}[ROI]{Region of Interest}
%\acro{rtc}[RTC]{Real Time Clock}
%\acro{saw}[SAW]{Surface Acoustic Wave}
\acro{smd}[SMD]{Surface-mounted device}
\acro{smt}[SMT]{Surface-mount technology}
\acro{sps}[Sps]{samples per second}
\acro{ssp}[SSP]{Streaming Serial Port}
\acro{sfdr}[SFDR]{Spurious-Free Dynamic Range}

\acro{snr}[SNR]{Signal-to-Noise-Ratio}
\acro{spi}[SPI]{Serial Peripheral Interface}
\acro{spp}[SPP]{Streaming Parallel Port}
%\acro{tqfp}[TQFP]{Thin Quad Flat Package}
\acro{ua}[u. a.]{unter anderen}
\acro{uat}[UAT]{User-Acceptance-Test}
\acro{usb}[USB]{Universal Serial Bus}
\acro{vhdl}[VHDL]{Very High Speed Integrated Circuit Hardware Description Language}
\acro{zb}[z.B.]{zum Beispiel}


\end{acronym}

%\begin{table}[!h]
%\caption{Abkürzungsverzeichnis}
%\label{Abkuerzungen}
%\begin{tabularx}{16 cm}{|l|X|}
%\hline
%\textbf{Abkürzung} & \textbf{Beschreibung}\\ \hline 
%a.h & ab hier \\\hline
%ADC & Analog-Digital Converter \\\hline
%\end{tabularx}

%\end{table}


\newpage
\section{Begriffsdefinitionen}\label{sec:begriffe}

\begin{longtable}{|p{5.5cm}|p{9.5cm}|}
KILLED & LINE!!!! \kill
\caption{Begriffsdefinitionen\label{Begriffe}}\\
\hline

\endfirsthead
\caption[]{(Fortsetzung Begriffsdefinitionen)}\\
\hline
\textbf{Begriff} & \textbf{Definition}\\ \hline \endhead
\textbf{Begriff} & \textbf{Definition} \\\hline

%\acf{alu} & Arithmetisch-logische Einheit welche ein elektronisches Rechenwerk in einen Prozessor abbildet. \\ \hline
\acf{api} & engl. Programmierschnittstelle, welche die wichtigsten Funktionen und Eigenschaften eines programmierten Moduls zu Verfügung stellt \\ \hline
\acf{db} & Das Dezibel ist eine nach Alexander Graham Bell benannte Hilfsmaßeinheit zur Kennzeichnung von Pegeln und Maßen (Logarithmische Größe). Diese Größen finden ihre Anwendung \ac{ua} in der Elektrotechnik sowie in der Akustik. \\ \hline
\ac{bom} & engl. Stückliste \\ \hline
%\ac{cad} & rechnerunterstütztes Designen \\ \hline
\acf{cw} Dopplerverfahren  & kontinuierliches Dopplerverfahren \\ \hline
\acf{fir} & Filter mit endlicher Impulsantwort oder Transversalfilter welcher ein diskreter, meist digitaler Filter darstellt\\ \hline
\ac{hal} & engl. Hardwareabstraktionsschicht \\ \hline
\ac{hf} & Frequenzen oberhalb hörbarer Schallwellen \\ \hline
\ac{ide} & engl. integrierte Entwicklungsumgebung \\ \hline
Interface & engl. Schnittstelle zwischen Funktionen oder Geräten \\ \hline
%\ac{ito} & Molekülverbindung aus Indium-Zinn-Oxid \\ \hline
\acf{ldo} & engl. linear Regler mit geringen Verlusten welcher auch mit geringfügig höheren Eingangsspannungen betrieben werden kann.\\ \hline
\ac{lna} & engl. rauscharmer Verstärker, welcher sich durch besondere Rauscharmut auszeichnet \\ \hline
\ac{pcb} & engl. Leiterplatte \\ \hline
\acf{pw} Dopplerverfahren & Gepulstes Dopplerverfahren \\ \hline
Ringing-Effekt	& Einschwingverhalten, welches durch die Steilheit der Flanken unstetiger Signale negativ beeinflusst wird. Dieser Effekt beruht auf dem Gibbssches Phänomen.  \\ \hline
\acf{roi} & engl. Bereich von Interesse \\ \hline
Schallimpedanz & Verhältnis von Schalldruck zu Schallschnelle \\ \hline
%\acf{sfdr} & Störungsfreier dynamischer Bereich - Abstand der größten Störung zur Grundschwingung in einem Spektrum anzugeben.\\ \hline
\acf{snr} & engl. Signal-Rausch-Verhältnis, beschreibt das Verhältnis zwischen Nutzsignal und des Rauschens in der Einheit \ac{db}\\ \hline
\acf{smd} & oberflächenmontiertes Bauelement auf Basis der \acf{smt} \\ \hline
%\acf{saw} & Akustische Oberflächenwelle \\ \hline
Transducer & engl. Schallkopf \\ \hline
\acf{uat} & Ist ein Test, welcher durch den Auftragnehmer durchgeführt wird, um den user die funktionale Sicherheit der Anwendung / des Produktes zu gewährleisten. \\ \hline
\end{longtable}