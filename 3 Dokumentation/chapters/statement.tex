\chapter*{Eidesstattliche Erklärung}
\addcontentsline{toc}{chapter}{Eidesstattliche Erklärung}
Hiermit versichere ich, Andreas Rehn, dass ich die vorliegende Arbeit selbstständig und ohne fremde Hilfe verfasst und keine anderen als die angegebenen Quellen oder Hilfsmittel verwendet habe. Alle Ausführungen, die fremden Quellen wörtlich oder sinngemäß entnommen wurden, sind als solche kenntlich gemacht.\\
Ich habe die Bedeutung der eidesstattlichen Versicherung und prüfungsrechtlichen Folgen (§26 Abs. 2 Bachelor-SPO \ac{bzw} §19 Abs. 2 Master-SPO der Hochschule der Medien Stuttgart) sowie die strafrechtlichen Folgen (siehe unten) einer unrichtigen oder unvollständigen eidesstattlichen Versicherung zur Kenntnis genommen.
\\
\\
\section*{Auszug aus dem Strafgesetzbuch (StGB)}
\textbf{§156 StGB} - Falsche Versicherung an Eides Statt\\
Wer von einer zur Abnahme einer Versicherung an Eides Statt zuständigen Behörde eine solche Versicherung falsch abgibt oder unter Berufung auf eine solche Versicherung falsch aussagt, wird mit Freiheitsstrafe bis zu drei Jahren oder mit Geldstrafe bestraft.
\\
\\
\\
\\
\begin{tabbing}
\jahr \= die die Breite der einzelne \= der einzelnen Spalten deklariert \kill %endet mit \kill
\jahr \> \> \fullname \\
\\
\\
\noindent\rule{\textwidth}{1pt}\\
Ort, Datum \> \> Unterschrift \\
\end{tabbing}